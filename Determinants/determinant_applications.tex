\documentclass[]{article}
\usepackage{graphicx}
\usepackage{amsmath,amsfonts,amssymb}
\usepackage[many]{tcolorbox}
\usepackage[%
margin=2cm,
includefoot,
bottom=2.55cm,
top=2.025cm,
headsep=0.5cm,
footskip=0.65cm
]{geometry}

\definecolor{myblue}{RGB}{0,46,142}

\newtcolorbox[auto counter]{mytheorem}[1][]{%
	enhanced jigsaw,
	colback=white,
	colframe=myblue,
	coltitle=myblue,
	fonttitle=\bfseries,
	sharp corners,
	detach title,
	enlarge left by=18mm,
	width=\linewidth-18mm,
	underlay unbroken and first={%
		\node[above,text=myblue,font=\bfseries,align=center] at ([xshift=-.5\textwidth,yshift=-7mm]interior.north) {\thetcbcounter};
	},
	breakable,
	pad at break=1mm,
	#1,
	code={\ifdefempty{\tcbtitletext}{}{\tcbset{before upper={\tcbtitle\par\medskip}}}},
}
\graphicspath{ {./images/} }


%opening
\title{L-20: Determinants and their Applications}
\author{Aahan Singh Charak\\Computer Science Grad}

\begin{document}
	\maketitle
	\section{Inverse Matrix}
	\vspace{10pt}
	
	For an n dimensional matrix, the formula for inverse can also be given by:\\
	
	\[
	A^{-1}=\frac{1}{\det(A)}*C^T
	\]\\
	
	\noindent
	Where, C is the matrix of co-factors.\\
	
	\noindent
	For a 2*2 matrix, the formula will be:\\
	
	\[
	A^{-1}=\frac{1}{ad-bc}\begin{bmatrix}
		d&-b\\
		-c&a
	\end{bmatrix}
	\]\\
	
	\vspace{10pt}
	
	\subsection{Why does this work?}
	\vspace{10pt}
	$A.A^T=I$\\
	
	\noindent
	$A.C^T=det(A).I$\\
	
	\[
\begin{bmatrix}
	a_{11}&a_{12}&...&a_{2n}\\
	a_{21}&a_{22}&...&a_{2n}\\
	.&.&.&.\\
	a_{n1}&a_{n2}&...&a_{nn}
\end{bmatrix}.\begin{bmatrix}
c_{11}&c_{21}&...&c_{n1}\\
c_{12}&c_{22}&...&c_{n2}\\
.&.&.&.\\
c_{1n}&c_{2n}&...&c_{nn}
\end{bmatrix}=\begin{bmatrix}
detA&...&0\\
0&detA&...\\
0&...&detA
\end{bmatrix}
	\]\\
	
\noindent
The first element of res matrix\\

\noindent
$detA_{11}=a_{11}c_{11}+a_{12}c_{12}+...+a_{1n}c_{1n}$ [This is the formula for determinant.(Cofactor formula)]\\

\noindent
Detailed proof can be seen from the book. Some information is mentioned in supplementary notes 2.\\

\vspace{10pt}

\section{Cramers rule}

\vspace{10pt}

$Ax=b$\\
$x=A_{-1}b$\\

\noindent
$x=\frac{1}{\det{A}}.C^Tb$\\

\noindent
According to cramer's rule, the individual components of x can be written as:\\

\noindent
$x_1=\frac{\det{B_1}}{\det{A}}$\\

\noindent
$x_2=\frac{\det{B_2}}{\det{A}}$\\

\noindent
$x_i=\frac{\det{B_i}}{\det{A}}$\\

\noindent
Where, $B_i$ is A with column i replaced by b.\\

\begin{mytheorem}[title=Note]
	Use elimination wherever possible as it is computationally less expensive than these formulas.
\end{mytheorem}

\vspace{10pt}

\section{Area and volumes of shapes}
These formulas and their proofs can be found in supplementary material 2 and the book.\\
\end{document}