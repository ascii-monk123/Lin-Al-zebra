\documentclass[]{article}
\usepackage{graphicx}
\usepackage{amsmath,amsfonts,amssymb}
\usepackage[many]{tcolorbox}
\usepackage[%
margin=2cm,
includefoot,
bottom=2.55cm,
top=2.025cm,
headsep=0.5cm,
footskip=0.65cm
]{geometry}

\definecolor{myblue}{RGB}{0,46,142}

\newtcolorbox[auto counter]{mytheorem}[1][]{%
	enhanced jigsaw,
	colback=white,
	colframe=myblue,
	coltitle=myblue,
	fonttitle=\bfseries,
	sharp corners,
	detach title,
	enlarge left by=18mm,
	width=\linewidth-18mm,
	underlay unbroken and first={%
		\node[above,text=myblue,font=\bfseries,align=center] at ([xshift=-.5\textwidth,yshift=-7mm]interior.north) {\thetcbcounter};
	},
	breakable,
	pad at break=1mm,
	#1,
	code={\ifdefempty{\tcbtitletext}{}{\tcbset{before upper={\tcbtitle\par\medskip}}}},
}
\graphicspath{ {./images/} }


%opening
\title{L-19: Determinant Formulas and Co-factor Matrix}
\author{Aahan Singh Charak\\Computer Science Grad}

\begin{document}
	\maketitle
	\section{Extended Formula}
	\vspace{10pt}
	
	\textbf{Method:} Take each row of matrix at a time and split row into no.of column pieces.(property 3b)\\
	
	\noindent
	\textbf{example} \\
	
	\[
	\begin{vmatrix}
		a&b\\
		c&d
	\end{vmatrix}=\begin{vmatrix}
	a&0\\
	c&0
\end{vmatrix}+ \begin{vmatrix}
a&0\\
0&d
\end{vmatrix} + \begin{vmatrix}
0&b\\
0&d
\end{vmatrix} + \begin{vmatrix}
0&b\\
c&0
\end{vmatrix}
	\]\\
	
	\noindent
	Only second and last determinants are useful as the rest have 0 columns.(singular matrix)\\
	
	\noindent
	Consider a 3*3 matrix with elements:\\
	
	\[
	\begin{bmatrix}
		a_{11} & a_{12} & a_{13}\\
		a_{21} & a_{22} & a_{23}\\
		a_{31} & a_{32} & a_{33}
	\end{bmatrix}
	\]\\
	
	
	\noindent
	For a 3*3 matrix 1st row will be divided into 3 pieces. For each of these pieces, 2nd row will give three pieces, therefore equating to 9 pieces. Similarily, considering third row, we get a total of 27 pieces. Explanation is in supplementary notes 1. However, out of these 27 pieces only some are useful. The useful determinants give the formula(6 total):\\
	
	\noindent
	$\mid A \mid = a_{11}(a_{22}a_{33}-a_{32}a_{23}) + a_{12}(a_{23}a_{31}-a_{21}a_{33}) + a_{13}(a_{33}a_{21}-a_{31}a_{22})$\\
	
	\vspace{10pt}
	
	\subsection{General Formula}
	\vspace{10pt}
	
	\noindent
	For 2*2 = 2 useful\\
	For 3*3 = 6 useful\\
	For n*n = n! useful\\
	
	\noindent
	$\det{A} = \sum_{r=1}^{n!} a_{1\alpha}a_{2\beta}a_{3\gamma}....a_{n\omega}$\\
	
	\noindent
	Where r=number of permutations and [$\alpha , \beta , \gamma, \delta$ etc. are permutations of 1 to nth column]\\
	
	\vspace{10pt}
	
	\section{Co-Factors}
	
	\vspace{10pt}
	
		$\mid A \mid = a_{11}(a_{22}a_{33}-a_{32}a_{23}) + a_{12}(a_{23}a_{31}-a_{21}a_{33}) + a_{13}(a_{33}a_{21}-a_{31}a_{22})$\\
		
	\noindent
	Here, the elements in parenthesis are co-factors.
	
	\begin{mytheorem}[title = General formula for co-factor]
		Cofactor of $a_{ij}$ = $C_{ij}$ = $\pm \det{(n-1 \ dimensional \ matrix \ with \ row \ i \ and \ column \ j \ erased)}$\\
			
			\[ 
		Where \ sign \ is,
		\begin{cases} 
			+ & if \ i+j=even\\
			- & if \ i+j=odd \\ 
		\end{cases}
		\]\\

	\end{mytheorem}

\vspace{10pt}

\subsubsection{Cofactor Formula for Determinants}
\vspace{10pt}

Determinant formula along ith row is given by(we can take along column too as transpose is equal in case of determinants.)\\

\noindent
$\det{A} = a_{i1}C_{i1}+a_{i2}C_{i2}+....+a_{in}C_{in}$\\


\begin{mytheorem}[title=Extra material]
	Extra material to develop more intuition can be found in book and supplementary notes 1.
\end{mytheorem}

	
	
	
\end{document}