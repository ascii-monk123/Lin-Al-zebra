\documentclass[]{article}
\usepackage{graphicx}
\usepackage{amsmath,amsfonts,amssymb}
\usepackage[many]{tcolorbox}
\usepackage[%
margin=2cm,
includefoot,
bottom=2.55cm,
top=2.025cm,
headsep=0.5cm,
footskip=0.65cm
]{geometry}

\definecolor{myblue}{RGB}{0,46,142}

\newtcolorbox[auto counter]{mytheorem}[1][]{%
	enhanced jigsaw,
	colback=white,
	colframe=myblue,
	coltitle=myblue,
	fonttitle=\bfseries,
	sharp corners,
	detach title,
	enlarge left by=18mm,
	width=\linewidth-18mm,
	underlay unbroken and first={%
		\node[above,text=myblue,font=\bfseries,align=center] at ([xshift=-.5\textwidth,yshift=-7mm]interior.north) {\thetcbcounter};
	},
	breakable,
	pad at break=1mm,
	#1,
	code={\ifdefempty{\tcbtitletext}{}{\tcbset{before upper={\tcbtitle\par\medskip}}}},
}
\graphicspath{ {./images/} }


%opening
\title{L-21: Eigenvalues and Eigenvectors}
\author{Aahan Singh Charak\\Computer Science Grad}

\begin{document}
	\maketitle
	\section{What are we looking at here?}
	\vspace{10pt}
	Consider a matrix A. Treat it like a function which takes in an input vector x. The function performs matrix multiplication and regurgitates something out. Suppose we get a multiple of the input vector as the output of the function. It would look something like this:\\
	
	\noindent
	$A.x=\lambda x$\\
	
	\noindent
	Here x is the eigenvector of matrix A and $\lambda$ is the eigenvalue for that eigenvector.\\
	
	\noindent
	We will now discuss some examples to understand these concepts:\\
	
	\vspace{10pt}
	
	\subsection{Singular matrix example}
    
	\vspace{10pt}
	
	If A is a singular matrix, there exists an element in the nullspace of A which is non-zero. x is one of those elements. The eigenvalue in this case will be $\lambda$=0.
	
	\vspace{10pt}
	
	\subsection{Projection matrix example}
	
	\vspace{10pt}
	
	Suppose A has a column space which is an n-dimensional plane. Projection matrix P projects on this plane. Any vector x in the plane will give the vector when fed to the projection matrix. $\therefore$ $P.x=1.x$.\\
	
	\noindent
	Here, eigenvector=x and eigenvalue=1.\\
	
	\noindent
	Vector perpendicular to the plane will be another eigen vector for P, as $P.x=0.x$\\
	
	\noindent
	Here, eigenvalue=0 and eigenvector=x.\\
	
	\begin{mytheorem}[title=Example 1]
		\[
		A=\begin{bmatrix}
			0&1\\
			1&0\\
		\end{bmatrix}
		\]\\
		
		\noindent
		\textbf{What are the eigenvalues and eigenvectors}
		
		\noindent
		\textbf{Sol: (a)} x=[1,1] is an evector and $\lambda =1$ is an evalue.
		
		\noindent
		\textbf{(b)} x=[-1,1] is an evector and $\lambda = -1$ is an evalue.
	
		
	\end{mytheorem}

\vspace{10pt}

\subsection{Relation between eigen values and trace of a matrix.}
\vspace{10pt}

$\sum{\lambda} = \sum{A_{ii}}$\\

\noindent
Where, i is the row number in an n*n dimensional squared matrix.\\

\vspace{10pt}

\section{Finding eigenvalues and eigenvectors}

\vspace{10pt}

\subsection{Eigenvectors}

\vspace{10pt}

\noindent
We know that,\\

\noindent
$Ax=\lambda x$\\

\noindent
$(A-\lambda I)x=0$\\

\noindent
If an eigenvector x exists, then A-$\lambda I$ must be singular. We can substitute $\lambda$ to get a solution.

\vspace{10pt}

\subsection{Eigenvalues}

\vspace{10pt}
Since $A-\lambda I$ is singular, we can simply equate the determinant to 0 in order to find eigenvalues.\\

\vspace{10pt}

\noindent
$\det{Singular}=0$\\

\noindent
$\det{A-\lambda I}=0$\\

\noindent
So, we first find eigenvalues and then find eigenvectors by substitution.\\

\vspace{10pt}

\section{A good property}

\vspace{10pt}

\noindent
If we add nI to A, $\lambda$ increases by n. Eigenvectors on the other hand, remain the same.\\

\noindent
\textbf{Proof:}\\

\noindent
If, $Ax=\lambda x$\\
Then, (A+nI)x=$\lambda x + nx$=($\lambda + n$)x.

\vspace{10pt}
\section{A property which doesn't exist tho.}
\vspace{10pt}

\noindent
Suppose, Ax=$\lambda x$, and\\
Bx=$\alpha x$\\

\noindent
But, (A+B)x=$(\lambda + \alpha)x$ does not always exist.\\

\noindent
x has to be the eigenvector for both matrices simultaneously for this property to exist. So, this property is not universal for all matrices A and B with eigenvectors and eigenvalues existing. Same can also be said for A.B\\

\noindent
More examples can be seen in supplementary notes 1.




	
\end{document}